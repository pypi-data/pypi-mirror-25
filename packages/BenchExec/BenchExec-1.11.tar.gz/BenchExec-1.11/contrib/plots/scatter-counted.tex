% LaTeX code for a scatter plot where color indicates frequency of data points
% Copy the tikzpicture environment to your own document
% and make sure the siunitx and pgfplots package are loaded,
% possibly with the options suggested here.
\documentclass{standalone}
\usepackage[
    group-digits=integer, group-minimum-digits=4, % group digits by thousands
    free-standing-units, unit-optional-argument, % easier input of numbers with units
    ]{siunitx}
\usepackage{pgfplots}
\pgfplotsset{
    compat=1.9,
    log ticks with fixed point, % no scientific notation in plots
    table/col sep=tab, % only tabs are column separators
    unbounded coords=jump, % better have skips in a plot than appear to be interpolating
    filter discard warning=false, % Don't complain about empty cells
    }
\SendSettingsToPgf % use siunitx formatting settings in PGF, too
% Define new pgfmath function for the logarithm to base 10 that also works with fpu library
\pgfmathdeclarefunction{lg10}{1}{\pgfmathparse{ln(#1)/ln(10)}}

\begin{document}

\begin{tikzpicture}
\begin{axis}[
    % axis labels
    xlabel=Value ``foo'' for Tool 1,
    ylabel=Value ``foo'' for Tool 2,
    % axis ranges
    xmin=0,
    xmax=100,
    ymin=0,
    ymax=100,
    domain=0:100,
    %
    clip mode=individual,
    axis equal image,
    colorbar,
    colormap/bluered,
    colorbar style = {
        ylabel=Number of results,
    },
    point meta min=0,
    point meta max=3,
    colorbar style={yticklabel=\pgfmathparse{10^\tick}\pgfmathprintnumber\pgfmathresult},
    ]
    \addplot+[mark=asterisk,scatter,scatter src=rnd*lg10(1000),only marks,samples=50]
        % The following line actually generates a random plot,
        % replace with something like the following for reading from CSV:
        %table[
        %     header=false,
        %     x index=2, % index of x column
        %     y index=6  % index of y column
        %     ] {scatter.counted.csv};
        (round(x),{round(x+rand*x)});
    \addplot[gray] {x};
\end{axis}
\end{tikzpicture}

\end{document}
